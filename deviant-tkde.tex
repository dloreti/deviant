\documentclass[10pt,journal,compsoc]{IEEEtran}


\usepackage{lineno,hyperref}
\modulolinenumbers[5]



% *** CITATION PACKAGES ***
%
\ifCLASSOPTIONcompsoc
  % IEEE Computer Society needs nocompress option
  % requires cite.sty v4.0 or later (November 2003)
  \usepackage[nocompress]{cite}
\else
  % normal IEEE
  \usepackage{cite}
\fi




\usepackage[font=footnotesize]{subfig, caption}
\usepackage{graphicx}
  \graphicspath{{./img/}}
  \DeclareGraphicsExtensions{.pdf,.jpeg,.png,.jpg}

\usepackage{todonotes}
\newcommand{\tododl}[1]{\todo[backgroundcolor=pink,size=\tiny]{DL: #1}}
\newcommand{\todoindl}[1]{\todo[inline,backgroundcolor=pink]{DL: #1}}
\newcommand{\todocdf}[1]{\todo[backgroundcolor=blue!20,size=\tiny]{CDF: #1}}
\newcommand{\todofc}[1]{\todo[backgroundcolor=yellow,size=\tiny]{FC: #1}}


\usepackage{xcolor}
\newcommand{\tcolor}[2]{\color{#1}{#2}\color{black}}

\usepackage{amsmath}

\usepackage{outlines}
\usepackage[nolist]{acronym}
\usepackage{amssymb}

\usepackage{algorithm}%,algorithmic}
\usepackage{algpseudocode}
\algdef{SE}[SUBALG]{Indent}{EndIndent}{}{\algorithmicend\ }%
\algtext*{Indent}
\algtext*{EndIndent}

\usepackage{listings}

\usepackage{amsthm}

\theoremstyle{definition}
\newtheorem{definition}{Definition}[section]
\newtheorem{theorem}{Theorem}[section]
\newtheorem{lemma}{Lemma}[section]
\newtheorem{remark}{Remark}[section]

\theoremstyle{plain}
\newtheorem{example}{Example}[section]



\usepackage[inline]{enumitem}

\usepackage{xfp}

\usepackage{booktabs}
\usepackage{multirow}
\usepackage{lmodern}
\usepackage{lmodern}
\usepackage[T1]{fontenc}


\usepackage{xspace}
\newcommand{\nd}{\texttt{NegDis}\xspace}
\newcommand{\declare}{\texttt{Declare}\xspace}
\newcommand{\rum}{\texttt{RuM}\xspace}
\newcommand{\declareminer}{\texttt{DeclareMiner}\xspace}
\newcommand{\decminer}{\texttt{DecMiner}\xspace}
\newcommand{\minclos}{\textsc{Cardinality}\xspace}
\newcommand{\subsetclos}{\textsc{Subset}\xspace}
\newcommand\dec[1]{$\mathsf{#1}$}
\newcommand\para[1]{\textnormal{\textsf{#1}}}
\newcommand\btext[1]{{\color{blue!30}{#1}}}

% MarcoM non condivide il termine choices, dice che è misunderstanding
% A tal scopo ho quindi creato una macro per deicedere che nome usare
% le sue proposte sono: negative witneses, discarders, rejecters, sheriffs marshals
\newcommand{\sheriff}{sheriffs}


%%%%%%%%%%%%%%%%%%%%%%%
%% Elsevier bibliography styles
%%%%%%%%%%%%%%%%%%%%%%%
%% To change the style, put a % in front of the second line of the current style and
%% remove the % from the second line of the style you would like to use.
%%%%%%%%%%%%%%%%%%%%%%%

%% Numbered
%\bibliographystyle{model1-num-names}

%% Numbered without titles
%\bibliographystyle{model1a-num-names}

%% Harvard
%\bibliographystyle{model2-names.bst}\biboptions{authoryear}

%% Vancouver numbered
%\usepackage{numcompress}\bibliographystyle{model3-num-names}

%% Vancouver name/year
%\usepackage{numcompress}\bibliographystyle{model4-names}\biboptions{authoryear}

%% APA style
%\bibliographystyle{model5-names}\biboptions{authoryear}

%% AMA style
%\usepackage{numcompress}\bibliographystyle{model6-num-names}

%% `Elsevier LaTeX' style
\bibliographystyle{elsarticle-num}

%%%%%%%%%%%%%%%%%%%%%%%%%%%%%%%%%%%%%%%%%%%%%%%%%%%%%%%%%%%%%%%%%%%%%%%%%%%%%%%%%%%%%%%%%%%%

\begin{document}


\title{Process discovery on deviant traces and other stranger things}

\author{Michael~Shell,~\IEEEmembership{Member,~IEEE,}
        John~Doe,~\IEEEmembership{Fellow,~OSA,}
        and~Jane~Doe,~\IEEEmembership{Life~Fellow,~IEEE}% <-this % stops a space
\IEEEcompsocitemizethanks{\IEEEcompsocthanksitem M. Shell was with the Department
of Electrical and Computer Engineering, Georgia Institute of Technology, Atlanta,
GA, 30332.\protect\\
% note need leading \protect in front of \\ to get a newline within \thanks as
% \\ is fragile and will error, could use \hfil\break instead.
E-mail: see http://www.michaelshell.org/contact.html
\IEEEcompsocthanksitem J. Doe and J. Doe are with Anonymous University.}% <-this % stops an unwanted space
\thanks{Manuscript received April 19, 2005; revised August 26, 2015.}}


% The paper headers
\markboth{Journal of \LaTeX\ Class Files,~Vol.~14, No.~8, August~2015}%
{Shell \MakeLowercase{\textit{et al.}}: Bare Demo of IEEEtran.cls for Computer Society Journals}


\IEEEtitleabstractindextext{%
\begin{abstract}
As the need to understand and formalise business processes into a model has grown over the last years, 
the process discovery research field has gained more and more importance, developing two different classes of approaches to model representation: procedural and declarative. 
%
Orthogonally to this classification, the vast majority of works envisage the discovery task as a one-class supervised learning process guided by the traces that are recorded into an input log. 

In this work instead, we focus on declarative processes and embrace the less-popular view of process discovery as a binary supervised learning task, where the input log reports both examples of the normal system execution, and traces representing ``stranger'' behaviours according to the domain semantics. We therefore deepen how the valuable information brought by both these two sets can be extracted and formalised into a model that is ``optimal'' according to user-defined goals. Our approach, namely \nd, is evaluated w.r.t. other relevant works in this field, and shows promising results as regards both the performance and the quality of the obtained solution.
\end{abstract}

% Note that keywords are not normally used for peerreview papers.
\begin{IEEEkeywords}
Computer Society, IEEE, IEEEtran, journal, \LaTeX, paper, template.
\end{IEEEkeywords}}


% make the title area
\maketitle


% To allow for easy dual compilation without having to reenter the
% abstract/keywords data, the \IEEEtitleabstractindextext text will
% not be used in maketitle, but will appear (i.e., to be "transported")
% here as \IEEEdisplaynontitleabstractindextext when the compsoc 
% or transmag modes are not selected <OR> if conference mode is selected 
% - because all conference papers position the abstract like regular
% papers do.
\IEEEdisplaynontitleabstractindextext
% \IEEEdisplaynontitleabstractindextext has no effect when using
% compsoc or transmag under a non-conference mode.



% For peer review papers, you can put extra information on the cover
% page as needed:
% \ifCLASSOPTIONpeerreview
% \begin{center} \bfseries EDICS Category: 3-BBND \end{center}
% \fi
%
% For peerreview papers, this IEEEtran command inserts a page break and
% creates the second title. It will be ignored for other modes.
\IEEEpeerreviewmaketitle



\IEEEraisesectionheading{\section{Introduction}\label{sec:introduction}}
% Computer Society journal (but not conference!) papers do something unusual
% with the very first section heading (almost always called "Introduction").
% They place it ABOVE the main text! IEEEtran.cls does not automatically do
% this for you, but you can achieve this effect with the provided
% \IEEEraisesectionheading{} command. Note the need to keep any \label that
% is to refer to the section immediately after \section in the above as
% \IEEEraisesectionheading puts \section within a raised box.

\IEEEPARstart{T}{his} demo file is intended to serve as a ``starter file''
for IEEE Computer Society journal papers produced under \LaTeX\ using
IEEEtran.cls version 1.8b and later.
% You must have at least 2 lines in the paragraph with the drop letter
% (should never be an issue)
I wish you the best of success.

\hfill mds
 
\hfill August 26, 2015








%%%\begin{frontmatter}
%%%
%%%\author[disi]{Federico Chesani}
%%%\author[fbk]{Chiara Di Francescomarino}
%%%\author[fbk]{Chiara Ghidini}
%%%\author[disi]{Daniela Loreti\corref{mycorrespondingauthor}}
%%%\cortext[mycorrespondingauthor]{Corresponding author}
%%%\ead{daniela.loreti@unibo.it}
%%%
%%%\author[unibz]{Fabrizio Maria Maggi}
%%%\author[disi]{Paola Mello}
%%%\author[unibz]{Marco Montali}
%%%\author[unibz]{Sergio Tessaris}
%%%
%%%\address[disi]{DISI - University of Bologna, Italy}
%%%\address[fbk]{Fondazione Bruno Kessler, Trento, Italy}
%%%\address[unibz]{Free University of Bozen/Bolzano, Italy}


%%%%%%%%%%%%%%%%%%%%%%%%%%%%%%%%%%%%%%%%%%%%%%%%%%%%%%%%%%%%%%%%%%%%%%%%%%%%%%%%%%%%%%%%%%%%


%%%\begin{abstract}
%%%As the need to understand and formalise business processes into a model has grown over the last years, 
%%%the process discovery research field has gained more and more importance, developing two different classes of approaches to model representation: procedural and declarative. 
%%%%
%%%Orthogonally to this classification, the vast majority of works envisage the discovery task as a one-class supervised learning process guided by the traces that are recorded into an input log. 
%%%
%%%In this work instead, we focus on declarative processes and embrace the less-popular view of process discovery as a binary supervised learning task, where the input log reports both examples of the normal system execution, and traces representing ``stranger'' behaviours according to the domain semantics. We therefore deepen how the valuable information brought by both these two sets can be extracted and formalised into a model that is ``optimal'' according to user-defined goals. Our approach, namely \nd, is evaluated w.r.t. other relevant works in this field, and shows promising results as regards both the performance and the quality of the obtained solution.
%%%\end{abstract}
%%%
%%%\begin{keyword}
%%%Process mining\sep process discovery\sep declarative process models\sep deviant traces
%%%\end{keyword}
%%%
%%%\end{frontmatter}

\linenumbers

% !TEX root = ../deviant.tex

\section{Introduction\tododl{DSS page limit: 34}}
\label{sec:intro}
The modelling of business processes is an important task to support decision-making in complex industrial and corporate domains. Recent years have seen the birth of the \ac{BPM} research area, focused on the analysis and control of process execution quality, and in particular, the rise in popularity of \emph{process mining} \cite{2012-Aalst}, which encompasses a set of techniques to extract valuable information from event logs. 
%
\emph{Process discovery} is one of the most investigated process mining techniques. It deals with the automatic learning of a process model from a given set of logged traces, each one representing the digital footprint of the execution of a case. 
%The model going to be learned must rely on a shared language to express the possible evolutions of business cases; just as the log file must have a clear, unambiguous syntax to express the relevant events occurred during the business process. 
Process discovery algorithms are usually classified into two categories according to the language they employ to represent the output model: procedural and declarative.
Procedural techniques envisage the process model as a synthetic description of all possible sequences of actions that the process accepts from an initial to an ending state. Declarative discovery algorithms---which represent the context of this work---return the model as a set of constraints equipped with a declarative, logic-based semantics, and that must be fulfilled by the traces at hand. 
%
Both approaches have their strengths and weaknesses depending on the characteristics of the considered process. For example, procedural techniques often produce  intuitive models, but may sometimes lead to ``spaghetti''-like outputs \cite{2009-Fahland, 2018b-Maggi}: in these cases declarative-based approaches might be preferable.
%Albeit extremely intuitive in some cases, procedural discovery may show poor results when the business process is unstructured and characterised by high variability \cite{2009-Fahland}. In that case, forcing the vision of the business process towards the template of a begin-to-end sequence of activities may result in a so-called ``spaghetti'' model \cite{2018b-Maggi}. 
%Declarative approaches are preferable in these situations because they represent the model as a simple elicitation of permitted and prohibited behaviours.

Declarative techniques rely on shared metrics to establish the quality of the extracted model, for example in terms of \emph{fitness}, \emph{precision}, \emph{generality}, and \emph{simplicity} \cite{2015-Adriansyah,2014-Broucke,2018-Ponce}. In particular, fitness and precision focus on the quality of the model w.r.t.~the log, i.e., its ability to accept desired traces and reject unlikely ones, respectively; generality measures the model's capability to abstract from the input by reproducing the desired behaviours, which are not assumed to be part of the log in the first place; finally, simplicity is connected to the clarity and understandability of the result for the final user. 

%The process discovery approaches can be also divided into two categories according to their vision on the model-extraction task.  As also pointed out by Ponce-de-Le\`on et al. \cite{2018-Ponce}, the vast majority of works %(e.g. \cite{2004-Aalst,2003-Weijters,2007-Gunther,2010-Aalst, altri!!!}) 
%can be seen as one-class supervised learning techniques, where the real-life log is analysed assuming it contains only \emph{positive} examples, i.e. information about the process instances that are allowed to take place. Some of the approaches in this category make use of thresholds and language biases to drive the discovery task, and extract valuable information about the occurrence of certain behavioural templates. 
Besides the declarative-procedural classification, process discovery approaches can be also divided into two categories according to their vision on the model-extraction task. 
As also pointed out by Ponce-de-Le\`on et al. \cite{2018-Ponce}, the vast majority of works in the process discovery spectrum (e.g. \cite{2004-Aalst,2003-Weijters,2007-Gunther,2010-Aalst}) can be seen as one-class supervised learning technique, while fewer works (e.g. \cite{2006-Maruster,2009-Goedertier,2009-Chesani}) intend model-extraction as a two-class supervised task---which is driven by the possibility of partitioning the log traces into two sets according to some business or domain-related criterion. Usually these sets are referred to as \emph{positive} and \emph{negative} examples \cite{2018-Ponce}, and the goal is to learn a model that characterises one set w.r.t. the other.

% su suggerimento di MarcoM
A further consideration stems from the \emph{completeness} of the log. Generally, a log contains and represents only a subset of the possible process executions. Other executions might be accepted or rejected from the viewpoint of the process, but this can be known only when a process model is learned or made available. This territory of \emph{unknown traces} will be ``shaped'' by the learned model, and more precisely by the choice of the discovery technique, and possibly by its configuration parameters. Approaches that consider positive examples only provide a number of heuristics, thus allowing the user to decide to which extent the yet-to-be-seen traces will be accepted or rejected by the discovered model---ranging from the extremity of accepting them all, to the opposite of rejecting them all.
The use of a second class of examples, identified on the basis of some domain-related criterion, allows to introduce some business-related criterion besides the heuristics.


%\todofc{Add a reference to the area of ``discriminative mining"?}
%availability of both \emph{positive} and \emph{negative} business executions, i.e. besides the allowed cases, these works assume that also examples of undesired behaviours can be exploited. %Inductive-learning techniques as those used in the works \cite{2009-Goedertier,2009-Chesani} for example, where a set of logical clauses is produced by the analysis of the input log, fall into this category.
%
%Clearly, the benefits of considering negative information can be significant as they allow to better control the degree of generalisation of the resulting model, as well as improve its simplicity by highlighting the parts of the model that are not significant to discriminate positive and negative behaviours. 
%Considering two classes of examples also allows  to better control the degree of generalisation of the resulting model, as well as to improve its simplicity by focusing on the model parts that allow to discriminate between the two sets. 


%The vast majority of works (e.g., \cite{2004-Aalst,2003-Weijters,2007-Gunther,2010-Aalst}) intend discovery as a classification task, where the set of traces in the input log must be analysed to extract valuable information about the occurrence of certain behavioural templates. This information is then used to build the execution model. Typically, the approaches in this category make use of thresholds, and language biases to drive the discovery task. 
%A smaller number of works \cite{2015-Guo,2007-Lamma,2009-Chesani} intend the model-extraction task as an inductive-learning process, where a set of logical clauses is produced by the analysis of the input log. 
%
%Both the categories have their advantages and shortcomings. Differently from induction-based techniques which do not usually stand out for their performance, classification-oriented discovery has reached high performance and effectiveness---provided that suitable metrics (e.g., constraint support, coverage, etc.) are defined to clearly assess the quality of the extracted model. 
%%
%Furthermore, while inductive-learning approaches have a solid theoretical background because inductive reasoning has been studied since the dawning of artificial intelligence, classification-oriented techniques do not seek perfection, but just a good approximation of the most common behaviours.
%Depending on the values assigned to parameters such as thresholds on support and coverage, classification-oriented approaches can provide completely different results. In general, the tuning of these parameters is not a straightforward task: the thresholds and language biases defined for a certain model extraction task, might not be suitable for a different use case. 
%
%On the other hand, inductive-learning approaches need both \emph{positive} and \emph{negative} examples to properly work, i.e. business execution cases that are compliant with the model going to be discovered are necessary as well as non-compliant cases. Classification-oriented discovery instead, works on positive examples only and discards as noise the negative ones, whenever they are present in the log. In particular, we can say that the availability of labelled positive and negative business execution examples is a crucially discriminative factor to opt for one process discovery view or the other.  
%%

%Actually, many studies endorse the thesis that, since in most of the real-life situations distinguish positive and negative cases in the input log is a very hard and time-consuming task, we should work as if the negative examples (which are assumed to be less frequent in the input) do not exist. Nonetheless, it is indisputable that, for each (meaningful) discovered business process model, there is a set of traces that are necessarily excluded because they are not compliant with the model. Such set constitutes a sort of ``upside-down world'', specular to the real world of positive, common and allowed cases. 


In this work, we focus on declarative process models expressed in the Declare language \cite{2008-Pesic}, and embrace the view of process discovery as a binary supervised learning task. Hence, our starting point is a classification of business traces into two sets, which can be driven by various motivations.
For example, in order to avoid underfitting and overfitting \cite{2010-Aalst}, many authors, as well as the majority of tools, suggest to ignore less-frequent traces (sometimes referred as \emph{deviances} from the usual behaviour \cite{2016-Nguyen}), thus implicitly splitting the log according to a frequency criterion. Another motivation for log partitioning could be related to the domain-specific need to identify ``stranger'' execution traces, e.g., traces asking for more (or less) time than expected to terminate, or undesirable traces that the user might want to avoid in future executions.

%, e.g.:
%\tcolor{red}{
%\begin{enumerate*}[label=(\textit{\roman*})]
%\item to avoid underfitting and overfitting \cite{2010-Aalst} many authors (as well as available tools) suggest to ignore \emph{deviant} \cite{2016-Nguyen}, less frequent traces, thus implicitly splitting the log; 
%\item domain-related criteria could be applied to identify ``stranger'' execution traces: a common case, for example, is to label as negative those execution traces that, to terminate, ask for more (or less) time than expected;
%\item the log might contain undesirable traces that the user might want to avoid in future business executions.
%\end{enumerate*}
%}

%For example, an approach would be to consider \emph{deviant} traces \cite{2016-Nguyen} as negative examples. Alternatively, domain-related criteria would allow to identify ``stranger'' execution traces: a common case, for example, is to label as negative those execution traces that, to terminate, ask for more (or less) time than expected.
%
Independently of the chosen criteria for splitting the log, we adopt the terms negative and positive example sets to identify the resulting partitioning, keeping in mind that the ``negative'' adjective is not necessarily connected to unwanted traces, but rather connected to a sort of ``upside-down world'' of ``stranger'' behaviours. The information carried by the negative example set diverges from that containing the positive examples but---coupled with it---can be used to understand the reasons why differences occur, ultimately providing a more accurate insight of the business process. 


%Independently of the chosen criteria for splitting the log, we envisage the negative example set as a sort of ``upside-down world'' of ``stranger'' behaviours, whose information can be coupled with the that of the positive example set, to understand the reasons why differences occur, and ultimately to provide a more accurate insight of the business process. 

For this reason, we hereby focus on learning a set of constraints that is able to reconstruct which traces belong to which set, while---whenever possible---reflecting the user expectations on the quality of the extracted model according to predefined metrics. 
%
In particular our approach, namely \nd, aims to discover a minimal set of constraints that allow to distinguish between the two classes of examples: in this sense, it can enrich existing approaches that instead provide richer descriptions of the positive example set onlyc. Indeed, our exploitation of the ``upside-down world'' is useful not only to better clarify what should be deemed compliant with the model and what should not, but also to better control the degree of generalisation of the resulting model, as well as to improve its simplicity.

%Our approach can be combined with other process discovery techniques as a post processing step to further refine and simplify complex models.
%
%Like the allowed traces can be exploited to extract information about the usual process model, we explore the possibility that the ``upside-down world'' of negative execution traces can be used---if made accessible---to understand the reasons why deviations from the common process model occur. This information is useful not only to better clarify what should be deemed compliant with the model and what should not, but also to specify parts of the business process in a more synthetic and effective way---by converting for example, a set of positive execution constraints into a single negative one.  
%To this end, we propose a novel vision on process discovery as a \emph{satisfiability problem}. 
%Our work, which envisages the process discovery task as a \emph{satisfiability problem}, can also be combined with other process discovery approaches as a post processing step to further refine and simplify complex models.

%\tcolor{blue}{This attempt yields ... \tododl{qualit\`a del nostro lavoro che emergono dagli esperimenti.}}


The contributions of our work can be listed as follows.
\begin{itemize}
\item A novel discovery approach, \nd, based on the underlying logic semantics of Declare, which makes use of the information brought by the positive and negative example sets to produce declarative models.
\item The adoption of a satisfiability-based technique to identify the models.
\item A heuristic to select the preferred models according to input parameters dealing with generalisation or simplicity.
\item An evaluation of the performance of \nd w.r.t. other relevant works in the same field.
\end{itemize}

%\todoindl{struttura dell'articolo?}
%Our work envisage the process discovery task as a \emph{satisfiability problem} and intertwines the constructive elements of both classification- and inductive-logic-oriented approaches into a single technique able to discover declarative process models by actively making use of both the positive traces and the ``upside-down world'' of deviant and negative examples---whenever they are available in the log. %\tododl{To say so, we'd need to put thresholds in the selection algorithm or propose an alternative version with thresholds. As for language bias, can we express it though the choice of $D$?}


%% !TEX root = ../deviant.tex

\section{Motivations}
\label{sec:motivations}
Process discovery focuses on the analysis of an event log in order to automatically learn the process model underpinning the cases of such input log. 
In general, the techniques in this field assume that the input log is not the complete elicitation of all the expected cases. Other traces not reported in there might be deemed compliant with the expected behaviour of the system. Therefore, the aim of process discovery techniques is necessarily to generalise the log by finding a compact way to express the usual behaviour of the systems, for example, by means of a structured or declarative process model. 
Such generalisation causes the resulting model to allow as compliant a larger set of traces w.r.t. the input log \cite{2011-Aalst}. This is one of the challenges of process discovery. If we could rely on logs reporting all expected behaviours, the extraction of the model would be a rather straightforward task, because we would need to learn a model exhaustively covering all the elicited cases. Also, if the model going to be learned is aimed for a following compliance checking task, model extraction would not be crucial, because a simple algorithm verifying the membership of a trace to the input set would serve the purpose.
On the other hand, in order to avoid the so-called ``spaghetti'' models, process discovery must also prevent overfitting \cite{2010-Aalst}. To this end, a widespread strategy is to overlook those process cases that show a particularly infrequent behaviour. A practice that is often obtained by checking that the extracted model constraints meet certain thresholds according to predefined metrics (e.g. fitness, precision, generality, and simplicity). 
It is therefore evident that, besides the usual attempt to generalise while extracting the model, process discovery necessarily performs an opposite attempt to increase the specificity by excluding some traces from the learning task.

The overlooked traces represent a sort of ``stranger'' behaviour, a deviation from the usual and expected conduct, which is the subject of \emph{deviance mining}\cite{2016-Nguyen}. 
%
Indeed, deviance mining is a field of process mining that encompasses techniques precisely to explain the reasons why a business process deviates from its normal execution. 
According to theory, deviations can have a positive or negative connotation. Positive deviations refer to desirable cases, where the business process shows particularly high performance, such as short execution time, low  cost, or particularly profitable outcomes \cite{2004-Spreitzer}. On the contrary, negative deviances usually refer to unwanted cases, that is situations non-conforming with the expected behaviour---because, for example, they produce an unwanted outcome or exceed the conventional execution time or cost \cite{2016-Nguyen}.
 
Most process discovery techniques do not consider negative examples. Indeed, a widespread position in this regard is that negative traces are usually not available. Event logs usually report a restricted number of non-compliant or unwanted case blurred in a much larger number of positive and more common examples. Nonetheless, we could say that even those process discovery techniques that are not explicitly based on the availability of negative example, actually make use of them in an implicit way, by assuming that they are somewhere present in the log and stating that they must be overlooked.
Differently from most previous approaches, we believe that negative (as well as positive) deviances might still have some informative content that is important to consider when discovering the process model. A more conscious shift is needed toward considering not only the positive and usual traces in the log, but also all the others, which provide information on negative and/or less frequent cases.

Another popular practice among process discovery techniques is that of defining a language bias, that clarifies the order in which the template of constraints must be considered during the learning task. Indeed, given a certain language to express the business model, different results may arouse depending from which patterns we search first among the log's cases. The choice of a language bias over another can be seen as a way to drive the discovery process towards a certain direction. Obviously, different results correspond to different ways to classify those traces that were not yet observed i.e., those case that are not present in the log, but might occur in the future.
In a sense, the language bias required by some process discovery techniques is a sort of implicit heuristic to decide in which order we must perform the generalisation or specialisation steps, while composing the business model.

Besides the benefit of taking into account deviant traces, we claim that process discovery should be able to take advantage from a more explicit heuristic to explore the search space. Among all possible behavioural patterns occurring in the logs, the choice of one over another should be driven by a more explicable strategy than simply defining an order of preferred constraints to be considered.
 
%\tcolor{blue}{On one hand, it is important to provide an easy way to label the traces in the log, so that positive and negative, usual and unusual cases are semantically separated and the information they carry can be exploited to refine the process model}\tododl{reword cause we do not focus on labelling. We need it.}. If negative examples are not present in the log, it is nonetheless important to clearly identify which traces are currently excluded by the discovered model because the characteristics of such excluded traces might still have significance for the model itself.
%On the other hand, the discovery process must operate taking into account all the traces in the log, extracting information from both common and less frequent traces, and providing a simple way to balance the discovered model between overfitting and underfitting. In order to achieve this, the discovery task will have to explore an extremely large space of possible models




% !TEX root = ../deviant.tex
\section{Background}%\tododl{A me piaceva di pi\`u Preliminaries...}}
\label{sec:back}

Our technique relays on the key concept of \emph{event log}, intending it as a set of observed business process executions, logged into a file in terms of all the occurred events. In this work, we adopt the \ac{XES} storing standard \cite{XES} for the input log. According to this standard, each \emph{event} is related to a specific process \emph{instance}, and describes the occurrence of a well-defined step in the process, namely an \emph{activity}, at a specific timestamp. The logged set of events composing a process instance is addressed as \emph{trace} or \emph{case}. 
From the analysis of the \emph{event log}, we want to extract a Declare \cite{2008-Pesic,2009-Aalst} \emph{model} of the business process.
Declare is one of the most used languages in process mining literature. Thanks to its declarative nature, it does not represent the process as a sequence of activities from a start to an end, but through a set of constraints, which can be mapped into \ac{LTL} formulae \cite{?}\tododl{cosa citiamo?}. These constraints must all hold true during the process execution.

Declare specifies a set of templates that can be used to model the process. 
A constraint is a concrete instantiation of a template involving one ore more process activities.
For example, the constraint \textsf{EXISTENCE(a)} is an instantiation of the template \textsf{EXISTENCE(X)}, and is used to specify that the activity \textsf{a} must occur in every trace; \textsf{INIT(a)} specifies that all traces must start with \textsf{a}. \textsf{RESPONCE(a,b)} imposes that if the \textsf{a} occurs, then \textsf{b} must follow. %\tododl{Necessaria notazione grafica in figura?}.
For a complete description of most used Declare templates see \cite{2008-Pesic}. 

We assume that the log contains both \emph{positive} traces---i.e., fulfilling all the constraints in the business model---and \emph{negative} traces---i.e., diverging from the expected behaviour by violating at least one constraint in the (intended) model. 

%We denote with $L^+$ the set of positive traces in the input event log, and with $L^-$ the set of the negative ones.


\paragraph{Language bias} Given a set of Declare templates $D$ and activities $A$, we identify with $D[A]$ the set of all possible grounding of templates in $D$ w.r.t. $A$, i.e. all the constraints that can be built using the given activities.

%\paragraph{Logs and traces} We assume that a trace $t$ is a \emph{finite} word over the set of activities (i.e.\ $t\in A^*$), and a log is a \emph{finite} set of traces. The consequences of this choice is that we don't consider parallel events and multiple occurrences of the same trace won't affect our discovery process.\todo{ST: this might be relaxed}
\paragraph{Logs and traces} We assume that a trace $t$ is a \emph{finite} word over the set of activities (i.e., $t\in A^*$, where $A^*$ is the set of all the words that can be build on the alphabet defined by $A$), and a log is a \emph{finite} set of traces. Multiple occurrences of the same trace won't affect our discovery process. We denote with $L^+$ and $L^-$ the sets of positive and negative traces, respectively, reported in the input event log. We assume that:
\begin{enumerate*}[label=(\textit{\roman*})]
\item $L^+ \cap L^- = \varnothing$, and 
\item for each trace $t \in L^-$ there exists at least a Declare constraint that allows to accept all the positive traces and exclude $t$.
\end{enumerate*}
In other words, we assume that the problem is feasible\footnote{Notice that sometimes real cases might not fulfil these assumptions. We will discuss this issue in section \ref{subsec:impl}}.

%\paragraph{Template subsumption} As pointed out by Di Ciccio et al. \cite{2017-DiCiccio} Declare templates can be organised into a subsumption hierarchy according to the logical implications that can be derived from their semantics. Intuitively, as the constraint \textsf{INIT(a)} forces all traces to start with \textsf{a}, this also implies that \textsf{a} must exist in every trace i.e., \textsf{EXISTENCE(a)}. This relation is valid irrespectively of the involved activity. In a sense, we could say that the template \textsf{EXISTENCE(X)} is \emph{more general} than \textsf{INIT(X)}. This idea is frequently expressed through the subsumption operator $\sqsupseteq$. Given two templates $d, d' \in D$, we say that $d$ \emph{subsumes} $d'$, i.e. $d$ \emph{is more general then} $d'$ (written $d\sqsupseteq d'$), if for any grounding of the involved parameters w.r.t. the activities in $A$, whenever a trace $t \in A^*$ is compliant with $d'$, it is also compliant with $d$ \cite{2017-DiCiccio} .

%\todo[inline]{ST: we don't use \emph{template subsumption} but the more general idea of generality of models (see \ref{def:generality}), even the rules are more general. Maybe this is not the right place for this paragraph, we should move it where we talk about the rules or to the related works section.}

%\tododl{aggiungere anche definizione di subsumption?}



% !TEX root = ../deviant.tex
\section{Preliminaries}
\label{sec:pre}

Our technique relays on the key concept of \emph{event log}, intending it as a set of observed business process executions, logged into a file in terms of all the occurred events. Each event is related to a specific process \emph{instance}, and describes the occurrence of a well-defined step in the process, namely an \emph{activity}. The logged set of events composing a process instance is addressed as \emph{trace} or \emph{case}. 
From the analysis of the \emph{event log}, we want to extract a Declare \cite{?} \emph{model} of the business process (or refine a preexisting one), able to represent the logged traces in a synthetic way, trough a set of constraints.
In particular, we assume some of the logged traces are \emph{positive}, i.e. they fulfil all the constraints in the business model, whereas others are \emph{negative} in the sense that they diverge from the expected behaviour by violating at least one constraint in the model. 
We denote with $L^+$ the set of positive traces in the input event log, and with $L^-$ the set of the negative ones.
We also consider a set $P$ of Declare constraints that are known to be valid on the positive traces. Such set can be the expression of domain knowledge or the result of a classification-oriented discovery algorithm previously applied to $L^+$. Obviously, $P$ can be also the empty set.

Given a set of Declare templates $D$ and activities $A$, we identify with $D[A]$ the set of all possible grounding of templates in $D$ w.r.t. $A$, i.e. all the constraints that can be built using the given activities. 
Our technique makes use of the concept of \emph{closure} to account for deduction over a set of Declare constraints.
%
For example, if we consider a set of constraints $D=\{\mathsf{INIT(a)}\}$, the closure of such simple set is $D=\{\mathsf{INIT(a), EXISTENCE(a)}\}$, because the fact that the process must start with an activity \textsf{a} implies that all process instances must contain \textsf{a}.
%
Since we do not have a complete calculus for the language of conjunctions of Declare constraints we impose just correctness; that is we require that for any closure ${cl}: 2^{D[A]} \rightarrow 2^{D[A]}$ the set of traces compliant with $C \subseteq D[A]$ is the same as the ones compliant with ${cl}(C)$, i.e.\tododl{siamo sicuri della doppia implicazione?}
\begin{equation}
\forall C,t  | C \subseteq D[A], t \models C \iff t \models {cl}(C)
\end{equation}


The goal of our technique is therefore to refine a previously learned (possibly empty) model $P$ by selecting a set of Declare constraints such that all positive traces and none of the negative are compliant\footnote{The conditions on all the positives and none of the negatives can be relaxed requiring a percentage of them; but this is outside the focus of the present work, and left for future investigation}, (where a trace is compliant with the set of constraints iff each constraint is satisfied by the trace).
Clearly, there can be several sets satisfying these conditions and we need to introduce a notion of fitness to select the preferred ones. 

In some context, \emph{generality} can be the fitness measure, that is we want to identify the set that is less committing in terms of restricting the admitted traces. In some other context on the contrary, we might be interested in the identification of a more \emph{specific} model. So besides allowing all traces in $L^+$ and forbidding all traces in $L^-$, the choice between a general or specific model, obviously affects the classification of the unknown traces.

Intuitively, a model $M$ is more general than another $M'$ if $M$ allows a superset of the traces accepted by $M'$, i.e. defining $\mathcal{C}_M$ the set of all traces compliant with $M$, 

\theoremstyle{definition}
\begin{definition}{\emph{Model generality/specificity}.}
A model $M$ is more general than another $M'$---and symmetrically $M'$ is more specific than $M$---if and only if $\mathcal{C}_{M'} \subset \mathcal{C}_M$.
\end{definition}

Obviously, testing the generality of a model according to this definition is not feasible, because it requires considering all the allowed/disallowed traces. The closure operator can be employed for such purpose. The two methods---comparing the set of traces $\mathcal{C}_M$ and $\mathcal{C}_{M'}$, or comparing their closures ${cl}(M)$ and ${cl}(M')$---are not equivalent because the deductive system deriving from the Declare language is not complete. Nonetheless, as the system is correct, the closure operator can be used to identify which model is more general/specific.

\section{The approach}
\label{sec:approach}

For the sake of modularity and easiness of experimenting with different hypotheses and parameters, we divide our algorithm into two clearly separate stages: one to identify the candidate models, and one to asses which is the best according the fitness measure we decide to apply. However these two steps can be merged into a single monolithic search-based algorithm.


As regards the first step, we are interested in the subsets $S$ of $D[A]$ satisfying the conditions:
\begin{enumerate}
\item $\forall t \in L^+, t \models S$, i.e. all positive traces are compliant with $S$;
\item $\forall t \in L^-, t \not\models (S \cup P)$, i.e. none of the negative traces is compliant with the union of $S$ and $P$.
\end{enumerate}
%
Algorithm \ref{alg:cand} reports the procedure to generate all possible candidate models satisfying these two conditions.
%
\begin{algorithm}
    \caption{Generation of all possible models allowing all traces in $L^+$ and disallowing at least one trace in $L^-$.}
    \label{alg:cand}
    %
    \textbf{Input:}  $D[A], L^+, L^-$\\
    \textbf{Output:} ${compatibles} \subseteq D[A]$, ${choices} : L^- \rightarrow 2^{D[A]}$
	%
	\begin{algorithmic}[1] 
   \Procedure{CandidateGeneration}{$D[A], L^+, L^-$} 
   	\State ${compatibles}= \{c \in D[A] | \forall t \in L^+ \models c\}$ \label{alg:cand:candidates}
	\For {$t \in L^-$}
		\State ${choices}(t) = \{c \in {compatibles} | t \not\models c\}$\label{alg:cand:choices}
	\EndFor
	\State \Return ${compatibles}, {choices}$
    \EndProcedure
    \end{algorithmic}
\end{algorithm}
%
It is implemented via a brute force strategy that first collects the set of all the constraints that are satisfied by all the positive traces (Line \ref{alg:cand:candidates}). Subsequently, each negative trace is associated (by means of the function ${choices}$) with the subset of those constraints that are not satisfied by the trace  (Line \ref{alg:cand:choices}). 
From the point of view of the implementation, the algorithm leverages the semantics of Declare patterns defined by means of regular expressions \cite{2017-DiCiccio} to verify the compliance of the traces. It is implemented in Go language employing a regexp implementation that is guaranteed to run in time linear in the size of the input\footnote{For more details, see the Go package regexp documentation at https://golang.org/pkg/regexp/}. 



Concerning the second step, we use an approximation algorithm (see Algorithm \ref{alg:select}) that makes use of the closure operator to guide the search of the optimal solution over the set of constraints. 
%
In practice, the algorithm uses the \ac{ASP} \cite{2008-Lifschitz} system clingo as an optimisation engine. The selection of the optimal stable model does not require the compliance verification on the traces, which are therefore not necessary as input. The candidate set and the positive discovery model are encoded as facts, while the closure operator is implemented as a set of \ac{ASP} rules.

\begin{algorithm}
    \caption{Selection of the best model according to custom model fitness.}
    \label{alg:select}
    %
    \textbf{Input:}  ${choices} : L^- \rightarrow 2^{D[A]}$, ${compatibles}$, $A, D[A], P, {cl}: 2^{D[A]} \rightarrow 2^{D[A]}$\\
    \textbf{Output:} $S \subseteq D[A]$
	%
	\begin{algorithmic}[1] 
   \Procedure{Selection}{${compatibles}, {choices},A, D[A], P, {cl}$} 
  	%\State $C = \{c \in D[A] ~|~ \exists t \in L^-, c \in {choices}(t)\}$
	\State \textbf{select} $S \subseteq {compatibles}$ \textbf{s.t.} \label{alg:subsetC}
	\Indent
		\State 1. $\forall t \in L^-, {choices}(t) \cap {cl}(S \cup P) \neq \varnothing$ 	\label{alg:select:1}
		\State 2. $is\_optimal(S)$									\label{alg:select:2}
		\State 3. $\not\exists S' \subset S ~|~ {cl}(S \cup P) = {cl}(S' \cup P)$		\label{alg:select:3}
	\EndIndent
	\State \Return $S$
    \EndProcedure
    \end{algorithmic}
\end{algorithm}\tododl{nella linea \ref{alg:subsetC} avrei potuto dire $S \subseteq C$ dove $C = \{c \in D[A] ~|~ \exists t \in L^-, c \in {choices}(t)\}$, sarebbe stata un'ottimizzazione, ma non so se sarebbe valida anche per la ricerca del modello massimale}

%The algorithm starts be defining the set $C$ of the constraints allowing all traces in $L^+$ and disallowing at least one trace in $L^-$.

The first condition of Algorithm \ref{alg:select} (Line \ref{alg:select:1}) specifies that $S$ must be a subset of ${compatibles}$ able to discard all traces in $L^-$ as non-compliant.

The second condition (Line \ref{alg:select:2}) accounts for the model fitness function, which can be customized according to the specific needs. For example, we could consider \emph{generality} as a fitness measure. 
%
%Intuitively, a model $M$ is more general than another $M'$ if $M$ allows a superset of the traces accepted by $M'$, i.e. defining ${comp(M)}$ the set of all traces compliant with $M$, $M$ is more general than $M'$---and symmetrically $M'$ is more specific than $M$---if ${comp}(M') \subset {comp}(M)$.\tododl{forse \`e meglio evidenziare questo paragrafo come una definizione?}

%Obviously, testing the generality of a model according to this definition is not feasible, because it requires considering all the allowed/disallowed traces. The closure operator can be employed for such purpose. The two methods (comparing the set of traces compliant with $M$ and $M'$, or comparing their closures ${cl}(M)$ and ${cl}(M)$) are not equivalent because the deductive system deriving from the Declare language is not complete. Nonetheless, as the system is correct, closure operator can be used to identify which model is more general/specific.
%
In that case, we could implement the $is\_optimal(S)$ function as a search for the \emph{minimal} $S$, intending it as the set $S$ of constraints for which there is no $S'$ such that ${cl}(S' \cup P) \subset {cl}(S \cup P)$.

On the other hand, if we want to find the most specific set of constraints---i.e. the model composed of Declare templates in $D$ that excludes the higher number of unknown traces---the $is\_optimal(S)$ operation must be implemented as a search for the \emph{maximal} $S$, intending it as the $S$ for which there is no $S'$ such that ${cl}(S \cup P) \subset {cl}(S' \cup P)$.

Finally, the last selecting condition (Line \ref{alg:select:3}) allows reducing the redundancy of the extracted model. This condition is desirable because---even when the user is interested in the most specific model---redundancy compromises the readability of the solution, without adding any value. Nonetheless, it is important to notice that the condition of Line \ref{alg:select:3} may not be sufficient to completely avoid redundancy because we do not have a complete calculus for the language of conjunction of Declare constraints.

Note that the set of activities $A$ is required as input to Algorithm \ref{alg:select} because the closure operator might generate constraints which range over all existing activities. For example, the constraint  $\textsf{INIT(a)}$ implies any constraint $\textsf{PRECEDENCE(a,X)}$ where \textsf{X} is an arbitrary activity.

Even if Algorithm \ref{alg:select} reduces the number of candidate solutions by excluding all those non fulfilling conditions 1., 2., and 3., it is not guaranteed to return a unique solution. If the number of solutions provided by the procedure is too high for human intelligibility, the optimality condition could be further refined by inducing a preference order in the returned solution. For example, one can be interested in being reported first the solutions with the lower number of constraints, or with certain Declare templates. The advantage of our approach is precisely in the possibility to implement off-the-shelves optimisation strategies, where---adapting the $is\_optimal(S)$ function or even the definition of the closure operator---the developer can easily experiment with different model fitness functions.

In order to better clarify the approach, we apply it to a very simple example.  
Consider the sets of positive and negative traces composed by only one trace each: $L^+=\{\textsf{bac}\}$ and $L^-=\{\textsf{ab}\}$. The alphabet of activities is clearly just $A=\{\textsf{a, b, c}\}$. Suppose we want to learn the most general model composed by only the Declare templates $D=\{\textsf{EXISTS, INIT}\}$\tododl{non ho inserito TRUE perch\`e non vorrei confondesse le idee quando si arriva a sceglie il set minimale}.

In this case, the set $D[A]$ can be easily elicited: $D[A]=\{ \textsf{EXISTS(a)}$, $\textsf{EXISTS(b)}$, $\textsf{EXISTS(c)}$, $\textsf{INIT(a)}$, $\textsf{INIT(b)}$, $\textsf{INIT(c)} \}$. Algorithm \ref{alg:cand} elects the following compatible constraints: ${compatibles}=$$\{ \textsf{EXISTS(a)}$, $\textsf{EXISTS(b)}$, $\textsf{EXISTS(c)}$, $\textsf{INIT(b)}\}$, and emits ${choices}(\textsf{ab})=$$\{ \textsf{EXISTS(c)}$, $\textsf{INIT(b)}\}$.

%Algorithm \ref{alg:select} easily computes the set $C$ of all constraints excluding at least one negate traces as $C=$$\{ \textsf{EXISTS(c)}$, $\textsf{INIT(b)}\}$. In this simple case, the subsets of $C$ satisfying the first condition (Line \ref{alg:select:1} of Algorithm \ref{alg:select}) are: $\{\textsf{EXISTS(c)}\}$, $\{\textsf{INIT(b)}\}$, and $\{\textsf{EXISTS(c)}$,$\textsf{INIT(b)}\}$. As we are interested in the most general model, both the solutions $S=$$\{\textsf{EXISTS(c)}\}$ and $S=$$\{\textsf{INIT(b)}\}$ are valid. Note that 

In this simple case, the subsets of ${compatibles}$ satisfying the first condition (Line \ref{alg:select:1}) of Algorithm \ref{alg:select} would be: 
\[
\begin{array}{cl}
 S_1= &  \{\textsf{EXISTS(c)}\}  \\
 S_2= &  \{\textsf{INIT(b)}\}   \\
 S_3= &  \{\textsf{EXISTS(c)}, \textsf{INIT(b)}\} \\
 S_4= & \{\textsf{EXISTS(c)}, \textsf{EXISTS(a)}\} \\
  ... & \\
 S_n= & \{\textsf{EXISTS(c)}, \textsf{INIT(b)}, \textsf{EXISTS(a)}\} \\
 ... &
\end{array}
\]

As we are interested in the most general model, both the solutions $S_1=$ $\{\textsf{EXISTS(c)}\}$ and $S_2=$$\{\textsf{INIT(b)}\}$ are valid. Note that these two solutions cannot be compared according to the definitions of generality or specificity because there exist traces (such as the unknown trace $\textsf{b}$) compliant with $S_2$ and non-compliant with $S_1$, i.e. there is no subset relation between $\mathcal{C}_{S_1}$ and $\mathcal{C}_{S_2}$.

On the contrary, if we are interested in the most specific set of constraints, the ${is\_optimal}(S)$ operation must return the maximal model $\{\textsf{EXISTS(a)}$, $\textsf{EXISTS(b)}$, $\textsf{EXISTS(c)}$, $\textsf{INIT(b)}\}$. Finally, the redundancy check operated by the third selecting condition discards the constraint $\textsf{EXISTS(b)}$, and correctly return the set $S=$$\{\textsf{EXISTS(a)}$, $\textsf{EXISTS(c)}$, $\textsf{INIT(b)}\}$. According to this model, the unknown trace \textsf{b} is negative and \textsf{bca} is positive. 


%
%\tcolor{blue}{**************************}
%
%Therefore, the algorithm is guided by the following schema:
%\begin{outline}
%\1 let the set of constraints ${compatibles}= \{c \in D[A] | \forall t \in L^+ \models c\}$;
%\1 let the function ${choices} : L^- \rightarrow 2^{D[A]}$ be such that ${choices}(t) = \{c \in {compatibles} | t \not\models c\}$;
%\1 search the sets $S \subseteq {compatibles}$ such that both the following conditions are satisfied:
%	\2 $\forall t \in L^-, {choices}(t) \cap {cl}(S \cup P) \neq \varnothing$;
%	\2 $S$ is minimal in the sense that there is no $S'$ such that $ {cl}(S' \cup P) \subset {cl}(S \cup P)$
%\end{outline}
%
%Clearly, any set generated by the algorithm satisfies the conditions 1 and 2 above but it might not be minimal. The requirement of minimality is guided by our need to determine the most general set of constraints.
%However, we also need to guarantee that no solution is missed, i.e., all possible solutions are included in at least one of the sets generated by the algorithm. The advantage of our proposal is that it can be implemented using off-the-shelves optimisation tools (as shown in the section below), where---adapting the closure operator and the optimisation conditions---we can easily experiment with different model fitness functions.
%
%Per ridurre ulteriormente il numero di S in output (oppure per ordinare le soluzioni trovate) potremmo contare gli elementi della closure. oppure puo` essere una preferenza tra template



% !TEX root = ../deviant.tex

\newcommand{\todofc}[1]{\todo[backgroundcolor=yellow,size=\tiny]{FC: #1}}
\newcommand{\todoinfc}[1]{\todo[inline,backgroundcolor=yellow]{FC: #1}}



\section{Experimental evaluation}
\label{sec:eval}

% Usually, an experimental evaluation of a process discovery algorithm would require to apply the proposed approach on one (or more) dataset, and to confront the learned model with those ones mined by other available approaches. However, our approach makes explicit use of information from traces labeled as ``negative'', while the totality of the approaches we could access have been designed to use positive traces only. Hence, confronting different approaches on different input data would not be fair.
% NO, questa motivazine non va bene...

One of the difficulties of evaluating process mining algorithms is that given a log, the underlying model might not be known before. As a consequence, it might be difficult to establish an \emph{ideal model} to refer and confront with. In this regard, a number of metrics and evaluation indexes have been proposed in the past to evaluate how a discovered model fits a given log \cite{2015-Adriansyah,2014-Broucke,2018-Ponce}. However, those metrics provide only a partial answer to the question of ``how good'' is the discovered model.\todofc{Frase molto forte, la vogliamo lasciare? Forse dovremmo almeno citare un paper che dica qualcosa di simile\ldots}
%
In the case of our approach, a further issue influences the evaluation process: the difficulty of performing a ``fair'' comparison with existing techniques, since the majority of the methods we could access have been designed to use ``positive'' traces only.


%A second issue is about the difficulty of confronting our approach with existing ones, especially if we consider that \todofc{Non so se questa cosa la voglio scrivere qui\ldots} our approach exploits information from traces labeled as ``negative'', while the majority of the approaches we could access have been designed to use positive traces only.

We pursued two different evaluation strategies. On one side, we defined a model, and from that model we generated a synthetic, artificial log, having care that it exhibits a number of desired properties: in a sense, this part of the evaluation can be referred as being about a ``controlled setting''. A first aim is to understand if our approach succeeds to discover a \emph{minimum} set of constraints for distinguishing positive from negative traces; a second aim is to qualitatively evaluate the discovered model, having the possibility to confront it with the original one. Experiments conducted on that synthetic log are reported and discussed in Section \ref{sec:syntheticlog}.

On the other side, we applied our discovery technique to some existing logs, thus evaluating it on some real data set. Again, this experiment has two aims: to understand weakness and strengths of our proposal w.r.t. to some relevant literature; and to confront the proposed approach with real-world data---and difficulties that real-world data bring along. Section \ref{sec:realdata} is devoted to present the selected logs and discuss the obtained results.

% Such aspect is further exacerbated by the fact that our approach exploits information from traces labeled as ``negative'', while the totality of the approaches we could access have been designed to use positive traces only.



\subsection{Experiments on a synthetic dataset}
\label{sec:syntheticlog}

% PER REFERENCE INTERBA NOSTRA:
%
% init(receive_loan_application)
% existence(assess_eligibility)
% precedence(assess_loan_risk, assess_eligibility)
% precedence(appraise_property, assess_eligibility)
% exclusive_choice(reject_application, send_acceptance_pack)
% precedence(assess_eligibility, reject_application)
% precedence(assess_eligibility, send_acceptance_pack)
% not_coexistence(reject_application, notify_approval)
% not_coexistence(receive_positive_feedback, receive_negative_feedback)
% exclusive_choice(send_acceptance_pack, receive_negative_feedback)
% precedence(appraise_property, assess_loan_risk)

% existence(0, 1, appraise_property)
% existence(0, 1, check_credit_history)
% existence(0, 1, check_income_sources)
% existence(0, 1, assess_loan_risk)
% , assess_eligibility % rimosso perché è già oggetto di un vincolo existence...
% existence(0, 1, reject_application)
% existence(0, 1, send_acceptance_pack)
% existence(0, 1, verify_receipt)
% existence(0, 1, cancel_application)
% existence(0, 1, notify_cancellation)
% existence(0, 1, approve_application)
% existence(0, 1, notify_approval)
% existence(0, 1, ask_for_customer_feedback)
% existence(0, 1, receive_positive_feedback)
% existence(0, 1, receive_negative_feedback)

The synthetic log has been generated starting from a Declare model, using the tool \cite{2020-Loreti}. The model has been inspired by the Loan Application process reported in \textcolor{red}{(2018, Dumas)}. \todofc{Se abbiamo bisogno di spazio, la descrizione in NL del processo può essere saltata, e rimandiamo il lettore direttamente al disegno.} In our model, the process starts when the \emph{loan application} is received. Before \emph{assessing the eligibility}, the bank proceeds to \emph{appraise the property} of the customer, and to \emph{assess the loan risk}. Then, the bank can either \emph{reject the application} or \emph{send the acceptance pack} and, optionally, \emph{notify the approval} (if not rejected). During the process execution the bank can also \emph{receive positive} or \emph{negative feedback} (but not both), according to the experience of the loan requester. It is not expected, however, that the bank receives a \emph{negative feedback} if the \emph{acceptance pack} has been sent. Moreover, due to temporal optimization, the bank requires that the \emph{appraise of the property} is done before \emph{assessing the loan risk}.
To ease the understanding of the loan application process, a Declare model of the process is reported in Fig. \ref{fig:ex}. Moreover, all the activities have been constrained to either not be executed at all, or to be executed at most once: in Declare terminology, all the activities have been constrained to \emph{absence2}.

\begin{figure}[t]
\centering
\includegraphics[width=0.6\columnwidth]{example1}
\caption{Loan approval declare process model }
\label{fig:ex}
\end{figure}

To test our approach, besides positive traces, we generated also negative traces. In particular, we generated traces that violate two different constraints:
% \todofc{Non ho messo il terzo test di violazione, perch\`e riguardava ancora una \emph{precedence}\ldots pensiamo se vogliamo aggiungerlo.}
\begin{enumerate}[label=(\alph*)]
\item the \emph{precedence(assess\_loan\_risk, assess\_eligibility)}, that is violated when either the latter activity is executed and the former is absent, or if both the activities appear in the log, but in the wrong temporal order;
%
\item the \emph{exclusive\_choice(send\_acceptance\_pack, receive\_negative\_feedback)}, that is violated when a trace either contains both the activities, or does not contain any of them.
\end{enumerate}
%
The resulting log consists of 64,000 positives traces, 25,600 traces that violate the constraint as in $(a)$, and 10,240 traces that violate the constraint as specified in $(b)$.
%
When fed with the positives traces and traces violating the constraint in $(a)$, our approach successfully manages to identify constraints that allow to clearly distinguish positives from negatives traces. Moreover, the discovered constraint coincides with the one we originally decided to violate during the generation phase. When confronted with the $(b)$ scenario, our approach again successfully managed to identify a minimum model able to discriminate between positive and negative traces, and the identified constraint is indeed logically consistent with the constraint originally selected for the violation.
% In particular, it is worthy to notice that our approach derived 
Table \ref{tab:syntResults} summarize the obtained results and reports the first selected model for each scenario.

% source of this data:
% file:///Users/federico/Google%20Drive/on-negative-traces/experiments/2020-07-14%20Federico/run_experiments_2020-07-30_130357_CEST.html
\todoinfc{I tempi di calcolo riportati in tabella potrebbero non essere giusti. Chiedere a Sergio\ldots in particolare, Sergio suggeriva che per gli step \texttt{compatibles} e \texttt{choices} dovremmo prendere i tempi \texttt{user + system}; per lo step \texttt{opt} invece nel caso di minclose dovremmo prendere la \texttt{CPUtime}, mentre per la subsetclose dovremmo prendere \texttt{CPUtime * (numconstraints bigger model) } }
\begin{table}
\tiny
\begin{tabular}{c c c c p{3cm} | p{3cm}}
\hline
Scenario & \#pos & \#neg & time & Originally violated constraint & First discovered model \\
\hline\hline
$(a)$ & 64,000 & 25,600 & \bf{Total: \fpeval{81.78 + 1.94 + 109.74 + 3.32 + 15.165}s} & \emph{precedence(assess\_loan\_ risk, assess\_eligibility)} & \emph{precedence( assess\_loan\_risk, assess\_eligibility)}\\
& & & Compatibles:  \fpeval{81.78 + 1.94}s  & & \\
& & & Choices:  \fpeval{109.74 + 3.32}s  & & \\
& & & Optimisation: 15.165s & & \\
%
\hline
%
$(b)$ & 64,000 & 10,240 & \bf{Total: \fpeval{81.78 + 1.94 + 94.51 + 2.96 + 1.379}s} & \emph{exclusive\_choice( send\_acceptance\_pack, receive\_negative\_ feedback)} & \emph{coExistence(reject\_application, receive\_negative\_feedback)}\\
& & & Compatibles:  \fpeval{81.78 + 1.94}s  & & \\
& & & Choices:  \fpeval{94.51 + 2.96}s  & & \\
& & & Optimisation: 1.379s & & \\
\hline
\end{tabular}
\caption{Models discovered when dealing with the synthetic data set.}
\label{tab:syntResults}
\end{table}

% PARAMETRI RuM usati:
% Templates: devi selezionarli tutti eccetto:
%  - not precedence
%  - not chain precedence
%  - not response
%  - not chain response
%  - not responded existence
% Constraint support: io ho provato 80,90 e 100 - qui penso che puoi vedere un po' in base ai risultati che ti escono
% Pruning types: come dicevo oggi li ho provati tutti e 4 ma secondo me e` sufficiente None.
% Vacuity detection: deve essere OFF
% Activity support filter: deve essere a 0%

For the sake of completeness, we decided to experiment also with the Process Discovery Tool of the Rum Framework\footnote{\url{https://rulemining.org/}}, that is based on the Declare Miner algorithm \cite{2018a-Maggi}. Based on the exploitation of positive traces only, Declare Miner discovers a rich model that describes as ``most exactly'' as possible the given traces. When fed with the positive traces of our artificial log, and with the \emph{coverage} parameter set to 100\% (i.e., prefer constraints that are valid for all the traces in the logs), the RuM Framework discover a model made of 514 constraints. If the coverage is relaxed to 80\% (prefer constraints that are satisfied by at least the 80\% of the traces), the model cardinality grows up to 1031 constraints.

In both cases the discovered model is able to distinguish between the positive and the negative traces. This is not surprising, since Declare Miner aims to identify all the constraints that hold for a given log: hence, it will discover also those constraints that allow to discern positive from negative traces. Rather, this result is a clear indication that indeed our artificial log has been constructed ``correctly'', since negative traces differ from positive ones for some specific constraints. This is typical of artificial logs, while real-life logs might not enjoy such property.
%
Another consideration is about the cardinalty of the discovered model: the Declare Miner approach provides a far richer description of the positive traces, at the cost perhaps of bigger models. Our approach instead has the goal of identifying the \emph{smallest} set of constraints that allow to discriminate between positive and negatives. In this sense, approaches like the one presented in this paper and Declare Miner are complementary.

% \todoinfc{ChiaraDFM, qui ci starebbe bene un paragrafetto in cui riportiamo i risultati ottenuti usando RuM. Il commento che potremmo scrivere e' che RuM trova un modello intero per coprire le positive, e riesce a distingure benissimo anche le negative: il risultato e' atteso, dato l'elevato numero di tracce che rende il synthetic data set molto informativo.}






\subsection{Evaluation on case studies from real data}
\label{sec:realdata}

\todoindl{Chiara DFM\&Sergio\&Fabrizio: descrizione dataset e risultati di esperimenti coi dati veri}

For the experimentation with real datasets, we used three real-life event logs: \textsc{cerv}, \textsc{sepsis} and \textsc{bpic12}. Starting from these event logs we generated 5 different datasets, each composed of a set of positive and a set of negative traces, by applying different criteria to distinguish between positive and negative traces, i.e., by labeling the event log with different labeling functions. 

\textsc{cerv} is an event log related to the process of cervical cancer screening carried out in an Italian cervical cancer screening center~\cite{2007b-Lamma}. Cervical cancer is a disease in which malignant (cancer) cells form in the tissues of the cervix of the uterus. The screening program proposes several tests in order to early detect and treat cervical cancer. It is usually composed by five phases: Screening planning; Invitation management; First level test with pap-test; Second level test with colposcopy, and eventually biopsy. The traces contained in the event log have been analyzed by a domain expert and labeled as compliant (positive traces) or non-compliant (negative traces) with respect to the cervical cancer screening protocol adopted by the screening center.

\textsc{sepsis}~\cite{Sepsis} is an event log that records trajectories of patients with symptoms of the life-threatening sepsis condition in a Dutch hospital.
%Œ\textsc{Sepsis} is an event log that records trajectories of patients with symptoms of the life-threatening sepsis condition in a Dutch hospital. 
Each case logs events since the patient’s registration in the emergency room until her discharge from the hospital. Among others, laboratory tests together with their results are recorded as events. The traces contained in the event log have been labelled based on their cycle execution time. In the \textsc{sepsis$_{mean}$} dataset, traces with a cycle time lower than the mean duration of the traces in the event log (\textasciitilde\xspace 28 days) have been labelled as positive, as negative otherwise. Similarly, in the \textsc{sepsis$_{median}$}, traces with a cycle time lower than the median duration of the traces in the event log (\textasciitilde\xspace 5 days)  have been labeled as positive, as negative otherwise.  

\textsc{bpic12}~\cite{BPIC2012} is a real-life event log pertaining to the application process for personal loans or overdrafts in a Dutch financial institute. It merges three intertwined sub-processes. Also in this case, the traces have been labelled based on their cycle execution time. In the \textsc{bpic12$_{mean}$} dataset (resp. \textsc{bpic12$_{mean}$}), traces with a cycle time lower than the mean (resp. median) duration of the traces in the event log (\textasciitilde\xspace 8 days, resp. \textasciitilde\xspace 19 hours) have been labelled as positive, as negative otherwise. 

%The \textsc{cerv} event log has been labeled based on the compliance of th


Table~\ref{tab:rl_datasets} summarizes the data related to the five resulting datasets.

\begin{table} [h]
	\centering
	\scalebox{0.8}{
		\begin{tabular} {l l c c c c c}
		\toprule
			\multirow{2}{*}{\textbf{Dataset}} & \multirow{2}{*}{\textbf{Log}} & \multirow{2}{*}{\textbf{Trace \#}} & \multirow{2}{*}{\textbf{Activity \#}} & \multirow{2}{*}{\textbf{Label}} & \textbf{Positive} & \textbf{Negative}  \\ 
			& & & & & \textbf{Trace \#} & \textbf{Trace \#} \\ \midrule
			\textsc{cerv$_{compl}$} & \textsc{cerv} & 157 & 16 & compliant & 55 & 102 \\ \midrule
			\textsc{sepsis$_{mean}$} & \multirow{2}{*}{\textsc{sepsis}} & \multirow{2}{*}{1000} & \multirow{2}{*}{16} & mean duration & 838 & 212 \\
			\textsc{sepsis$_{median}$} &  &  &  & median duration & 525 & 525 \\ \midrule
			\textsc{bpic12$_{mean}$} & \multirow{2}{*}{\textsc{bpic12}} & \multirow{2}{*}{13087} & \multirow{2}{*}{36} & mean duration & 8160 & 4927  \\
			\textsc{bpic12$_{median}$} &  &  &  & median duration & 6544  & 6543 \\ 			
			\bottomrule
		\end{tabular}}
		\caption{Dataset description}
		\label{tab:rl_datasets}
\end{table}

The results obtained by applying the \nd algorithm are summarised in Table~\ref{tab:rl_results}. The table reports for each dataset, (i) the results related to the \subsetclos (Eq. \ref{eq:most-gen} and Eq.\ref{eq:redun} in Section~\ref{sec:approach}) and \minclos (Eq. \ref{eq:simpl1} and \ref{eq:simpl2} in Section~\ref{sec:approach}) closure - in terms of number of returned models\footnote{We stop generating models after $20$ models, i.e., \para{max} in Table~\ref{tab:rl_results} indicates that more than $20$ models have been returned.}, in terms of minimum size of the returned models, as well as in terms of percentage of negative traces violated by the returned model. Moreover, the table reports the time required for computing the set of compatibles, the set of choices, as well as for the \subsetclos and \minclos closures.

\begin{table} [h]
	\centering
	\scalebox{0.58}{
		\begin{tabular} {l | c c c | c c c | c c c c}
		\toprule
			\multirow{4}{*}{\textbf{Dataset}} & \multicolumn{3}{c|}{\textbf{\subsetclos}} &  \multicolumn{3}{c|}{\textbf{\minclos}}  & 	\multicolumn{4}{c}{\textbf{ Required Time (s)}} \\ \cmidrule{2-11}
			& \textbf{Number} & \textbf{Min} & \textbf{Violated} & \textbf{Number} & \textbf{Min} & \textbf{Violated} & \multirow{3}{*}{\textbf{Comp.}} & \multirow{3}{*}{\textbf{Choices}} & \multirow{3}{*}{\textbf{\subsetclos}} & \multirow{3}{*}{\textbf{\minclos}}  \\			
			& \textbf{of} & \textbf{model} & \textbf{$L^{-}$} & \textbf{of} & \textbf{model} & \textbf{L$^{-}$} & & & & \\
			& \textbf{models} & \textbf{size} & \textbf{trace \%} & \textbf{models} & \textbf{size} & \textbf{trace \%} & & & & \\\midrule
			\textsc{cerv$_{compl}$} & \para{max} & 4 & 100\% & \para{max} & 4 & 100\% & 0.12 & 0.33 & 0.065 & 0.045\\ \midrule
			\textsc{sepsis$_{mean}$} & 1 & 8 & 4.25\% & 1 & 8  & 4.25\% & 0.73 & 1.04 & 0.039 & 0.035\\ 
			\textsc{sepsis$_{median}$} & \para{max} & 14 & 26.86\% & 16 & 14 & 26.86\% & 0.45 &	1.5 & 0.2 & 0.087 \\ \midrule
			\textsc{bpic12$_{mean}$} & \para{max} & 12 & 1.42\% & \para{max} & 12 & 1.42\% & 13.51 & 31 & 0.096 & 0.066  \\ 
			\textsc{bpic12$_{median}$} & \para{max} & 23 & 36.59\% & \para{max} & 22 & 36.59\% & 13.32 & 37.63 & 359.164 & 43.846 \\ 			
			\bottomrule
		\end{tabular}}
		\caption{\nd results on the real-life logs}
		\label{tab:rl_results}
\end{table}

The table shows that for the \textsc{cerv$_{compl}$} dataset, \nd is able to return models that satisfy the whole set of positive traces and violate the whole set of negative traces (the percentage of violated traces in $L^-$ is equal to 100\%) with a very low number of constraints (4). For the other datasets, the returned models are always able to satisfy all traces in $L^+$, however not all the negative traces are violated by the returned models. In case of the datasets built by using the mean of the trace cycle time, the percentage of violated traces is relatively small ($4.25\%$ for \textsc{sepsis$_{mean}$} and $1.24\%$ for \textsc{bpic12$_{mean}$}), as the number of constraints of the returned models ($8$ for \textsc{sepsis$_{mean}$} and $12$ for \textsc{bpic12$_{mean}$}). Nevertheless, \nd is able to obtain reasonable results with the real life datasets built with the median of the trace cycle time. Indeed, it is able to identify $14$ (resp. $22$-$23$) constraints able to accept all traces in $L^+$ and to cover about $27\%$ (resp. $37\%$) of the traces in $L^-$ for \textsc{sepsis$_{median}$} (resp. \textsc{bpic12$_{median}$}). The difference in terms of results between the \textsc{cerv$_{compl}$} and the other datasets is not surprising. Indeed, while what characterizes positive and negative traces in the \textsc{cerv$_{compl}$} dataset depends upon the control flow (i.e., it depends on whether each execution complies with the cervical cancer screening protocol adopted by the screening center), when mean and median cycle time are used, the difference between positive and negative traces could likely not exclusively depend upon the control flow of the considered traces. Overall, the inability to identify a set of constraints that is able to fulfil all traces  in $L^+$ and to violate all negative ones is due to a bias of the considered language (\declare without data) that does not allow to explain the positive traces without the negative ones.

The difference of the results obtained with the mean and the median cycle time can also be explained as a language bias issue for the specific labelled datasets. Indeed, while when the positive and negative trace sets are quite balanced (i.e., for \textsc{sepsis$_{median}$} and \textsc{BPIC12$_{median}$}) \nd is able to identify a set of constraints (related to the control flow) describing the traces with a low-medium cycle time and excluding the ones with a medium-high cycle time, when the sets of the positive and the negative traces are quite imbalanced (i.e., for \textsc{sepsis$_{mean}$} and \textsc{BPIC12$_{mean}$})
%(the size of the set of the positive traces is about 4 times, resp. 2 times, the size of the set of the negative traces in \textsc{sepsis$_{mean}$} and \textsc{BPIC12$_{mean}$}, respectively)
 characterizing the high number of traces with a low or medium cycle time while excluding the ones with a very high cycle time can become hard. 

The table also shows that \nd is overall very fast for small datasets (e.g., less than one minute for \textsc{cerv$_{compl}$}), while it requires some more time for large ones (e.g., \textsc{bpic12$_{mean}$} and \textsc{bpic12$_{median}$}). While the time required for computing \textit{compatibles} and \textit{choices} seems to be related to the size of the dataset, the time required for computing the closures seems to depend also on other characteristics of the datasets. 

%We compared the obtained results with the ones obtained with state-of-the-art techniques for the discovery of declarative models starting (i) from  only positive traces; and (ii) from both positive and negative traces. 

%For the former case, we used the state-of-the-art Declare miner algorithm~\cite{2018a-Maggi}  implemented in the RuM toolkit~\cite{2020-Alman}. For the latter we compared the results obtained with \nd with the ones of DecMiner, the approach based on Inductive Logic Programming proposed in~\cite{2007b-Lamma}.

Compared to state-of-the-art techniques for the discovery of declarative models starting from the only positive traces, \nd is able to return a small number of constraints satisfying all traces in $L^+$ without decreasing the percentage of violated traces in $L^-$.  
%Concerning the classical declarative discovery from only positive execution traces, 
Among the classical declarative discovery approach, we selected the state-of-the-art \declareminer algorithm~\cite{2018a-Maggi} implemented in the \rum toolkit~\cite{2020-Alman}. We discovered the models using the only positive traces and setting the \para{support} parameter, which measures the percentage of (positive) traces satisfied by the \declare model, to 100\%\footnote{We run the \declareminer algorithm with vacuity detection disabled, \para{activity support filter} set to 0\%, using both transitive closure and hierarchy-based reduction of the discovered constraints, as well as with all the Declare templates, except for \dec{NotResponse}, \dec{NotPrecedence}, \dec{NotChainPrecedence}, \dec{NotChainResponse}, \dec{NotRespondedExistence}.}
%We used the \declareminer algorithm\footnote{We run the \declareminer algorithm with vacuity detection disabled, \par{activity support filter} set to 0\%, using both transitive closure and hierarchy-based reduction of the discovered constraints, as well as with all the Declare templates, except for \dec{NotResponse}, \dec{NotPrecedence}, \dec{NotChainPrecedence}, \dec{NotChainResponse}, \dec{NotRespondedExistence}.} to discover the \declare model starting from the only positive traces with different values of the \para{support} parameter, which measures the percentage of (positive) traces satisfied by the \declare model.  \todocdf{Move this part when the results are presented for the synthetic data}

Table~\ref{tab:rl_declare_miner} summarizes the obtained results. The table reports for each dataset, the size of the model in terms of number of constraints, as well as the percentage of negative traces violated by the model. For lower values of the \para{support} parameter, i.e., for a lower percentage of positive traces satisfied by the model, the model returned by the \declareminer violates a higher percentage of negative traces. In this way the \para{support} parameter allows for balancing the percentage of positive trace satisfied and negative traces violated. 

As hypothesized, the optimization mechanism in \nd is able to identify a small set of constraints, that guarantees the satisfaction of all traces in $L^+$ and the same percentage of negative trace violations obtained with \declareminer (with \para{support} to 100\%).

\begin{table} [ht]
	\centering
	\scalebox{0.8}{
		\begin{tabular} {l c c}
		\toprule
			\textbf{Dataset} & \textbf{Model size} & \textbf{Violated L$^{-}$ trace \%}  \\ \midrule
			\textsc{cerv$_{compl}$} & 323 & 100\% \\ \midrule
			\textsc{sepsis$_{mean}$} & 210 & 4.25\%\\ 
			\textsc{sepsis$_{median}$} & 202 & 26.86\% \\ \midrule
      \textsc{bpic12$_{mean}$} & 514 & 1.42\% \\ 
			\textsc{bpic12$_{median}$} & 532 & 36.59\% \\ 
		\bottomrule
		\end{tabular}}
		\caption{\declareminer results}
		\label{tab:rl_declare_miner}
\end{table}

Finally, we evaluated the results obtained with \nd relying on the same procedure and dataset (\textsc{cerv$_{compl}$}) used in ~\cite{2007b-Lamma} to assess the results of \decminer, a state-of-the-art declarative discovery approach based on Inductive Logic Programming that is able to use both positive and negative execution traces. 
%To this aim, we run the same procedure described in~\cite{2007b-Lamma} on the same dataset (the \textsc{cerv$_{compl}$} dataset) used in the same paper. 
Five fold-cross validation is used, i.e., the \textsc{cerv$_{compl}$} dataset is divided into 5 folds and, in each experiment, 4 folds are used for training and the remaining one for validation purposes. The average \emph{accuracy} of the five executions is collected, where the accuracy is defined as the sum of the number of positive (compliant) traces that are (correctly) satisfied by the learned model and the number of negative (non-compliant) traces that are (correctly) violated by the learned model divided by the total number of traces. 
%\begin{equation*}
% accuracy = (|satisifed L^+ | + |violated L^-|)/ |L^+ + L^-|
%\end{equation*}

Table~\ref{tab:acc_results} reports the obtained accuracy values for the \decminer, the \declareminer (with the \para{support} parameter set to 100\%) and the \nd (both for the \minclos and \subsetclos closure) approach. The table shows that on this specific dataset, \nd and \decminer have very close performance (with \nd \minclos performing slightly better). \declareminer presents instead on average a slightly lower accuracy mainly due to the highly constrained discovered model that, on the hand, allows for violating all negative traces in the validation set, and, on the other hand, leads to the violation of some of the positive traces in the validation set.

\begin{table} [h]
	\centering
	\scalebox{0.8}{
		\begin{tabular} {l c}
			\toprule
			\textbf{Approach} & \textbf{Accuracy} \\ \midrule
			 \decminer &  97.44\%\\ \midrule
			 \declareminer & 96.79\%\\ \midrule
			 \nd (\subsetclos) & 97.38\% \\ 			
			 \nd (\minclos) & 97.57\% \\ 
			\bottomrule
		\end{tabular}}
		\caption{Accuracy results obtained with \declareminer, \decminer and \nd}
		\label{tab:acc_results}
\end{table}





\input{sections/related}
%% !TEX root = ../deviant.tex

\section{Discussion}
\label{sec:discuss}\tododl{x Dani: sezione da raffinare alla fine. Molte riflessioni penso non abbiano pi\` u senso.}
Our work is focused on declarative process models. However, it is worth to underline that a declarative process discoverer taking advantage of explicitly defined positive and negative examples is not necessarily an alternative to procedural discovery techniques. 
Indeed in some cases, when correct thresholds and language biases are adopted, procedural discoverers have the great advantage to provide the user with a rather easy-to-understand definition of the process model. Nonetheless, the informative content provided by those process cases that are discarded by procedural discoverer (e.g., in order to avoid spaghetti models) can still be extremely important. 
As our approach extracts valuable information from $L^-$ traces without excluding any trace of $L^+$, it could also be applied as a post processing technique to enrich the output of procedural discoverers with declarative constraints.
%A declarative process discoverer taking advantage of explicitly defined positive and negative examples can extract valuable information from such discarded traces and synthesize it into declarative constraints. 
The resulting output would be an hybrid procedural/declarative process model, showing a simple and handy structured representation of the main business process together with a set of declarative constraints. The goal of such constraints would be to account for less frequent deviances and prohibited behaviours in a much more synthetic and easy-to-understand way with respect to an equivalent spaghetti-like procedural formulation.

Furthermore, such hybrid solution could also greatly simplifying the elicitation of long-term dependencies between activities that occur at the beginning of the process and those carried out towards the end. Indeed, the structured nature of procedural approaches makes them not properly suitable to express such dependencies.
One current way to tackle such issue is through the employment of global variables and if statements to control the execution flow of each instance.
Kalenkova et al. \cite{2020-Kalenkova} propose e process discovery technique devoted to repair free-choice procedural workflows with additional modelling constructs, which can more easily capture non-local dependencies. Nonetheless, since such additional constraints are intended to preserve the procedural nature of the model, the result may increase its complexity and ultimately affect its readability.
An hybrid procedural/declarative model formulation would maintain a structured form to express the model while integrating it with handy declarative long-term constraints involving activities occurring far from each other in the workflow.
%
%For example, consider the load application process depicted in Fig. ... The procedural nature of the model makes it particularly easy to understand for a human subject. Nonetheless, if we want to add a rather simple constraint such as: "Do not ask customer feedback if the application was cancelled", a substantial modification of the model is required.
%Indeed, the model must state that the branch including "Cancel application" (and then "Notify cancellation") must be followed by the END event, whereas any other branch can still include "Ask for customer feedback" before the END event.
%In practice, adding a constraint of such kind force us to add a alternative branch towards the end of the model.
%If many alternative paths are present in the model and we need to add many conditions of this kind, the diagram in Fig.. may quickly turn into a spaghetti model.
%A more compact and readable way to apply this modification is to maintain the present process model structure and equipping it with a declarative elicitation of the prohibited behaviours. In the considered case, the simple inclusion of a constraint such as "NOT PRECEDENCE(Cancel application, Ask for customer feedback)" prevents all forbidden paths. 
%
This idea of a hybrid procedural/declarative model formulation has been explored by various works and proved to be particularly effective in the field of medical clinical guidelines \cite{2009a-Bottrighi,2009b-Bottrighi, 2011-Bottrighi}. A wider landscape of applications is considered by Maggi et al. in the work \cite{2018b-Maggi}. %The technique starts from a structured business process model and adopts non-deterministic finite-state automaton manipulation to detect violations of compliance requirements expressed as temporal declarative rules.

\todoindl{Riflessione su: Our goal is admitting all cases in $L^+$ and discarding all those in $L^-$. These conditions can be relaxed by requiring only a percentage of them?thus affecting the fitness and precision of the resulting model.}

% !TEX root = ../deviant.tex

\section{Conclusion}




%\section*{References}

\bibliography{deviant}
% !TEX root = ../deviant.tex

\begin{acronym}[PUB/SUB]
    \acro{ASP}{Answer Set Programming}
    \acro{BPM}{Business Process Management}
    \acro{ILP}{Inductive Logic Programming}
    \acro{XES}{eXtensible Event Stream}
\end{acronym}

\end{document}