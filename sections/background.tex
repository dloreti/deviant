% !TEX root = ../deviant.tex
\section{Background}%\tododl{A me piaceva di pi\`u Preliminaries...}}
\label{sec:back}

Our technique relies on the key concept of \emph{event log}, intending it as a \emph{set} of observed process executions, logged into a file in terms of all the occurred events.
%
% Other approaches adopt view the log as a multi-set
%
In this work, we adopt the \ac{XES} storing standard \cite{XES} for the input log. According to this standard, each \emph{event} is related to a specific process \emph{instance}, and describes the occurrence of a well-defined step in the process, namely an \emph{activity}, at a specific timestamp. The logged set of events composing a process instance is addressed as \emph{trace} or \emph{case}. 
From the analysis of the \emph{event log}, we want to extract a Declare \cite{2008-Pesic,2009-Aalst} \emph{model} of the process.
Declare is one of the most used languages for declarative process modeling. Thanks to its declarative nature, it does not represent the process as a sequence of activities from a start to an end, but through a set of constraints, which can be mapped into \ac{LTL} formulae over finite traces \cite{DBLP:journals/tweb/MontaliPACMS10,DBLP:conf/ijcai/GiacomoV13}. These constraints must all hold true when a trace complete.

Declare specifies a set of templates that can be used to model the process. 
A constraint is a concrete instantiation of a template involving one ore more process activities.
For example, the constraint \textsf{EXISTENCE(a)} is an instantiation of the template \textsf{EXISTENCE(X)}, and is used to specify that activity \textsf{a} must occur in every trace; \textsf{INIT(a)} specifies that all traces must start with \textsf{a}. \textsf{RESPONSE(a,b)} imposes that if the \textsf{a} occurs, then \textsf{b} must follow, possibly with other activities in between. %\tododl{Necessaria notazione grafica in figura?}.
For a description of the most common Declare templates see \cite{2008-Pesic}. 

We assume that the log contains both \emph{positive} traces---i.e., satisfying all the constraints in the business model---and \emph{negative} traces---i.e., diverging from the expected behaviour by violating at least one constraint in the (intended) model. 

%We denote with $L^+$ the set of positive traces in the input event log, and with $L^-$ the set of the negative ones.


\paragraph{Language bias} Given a set of Declare templates $D$ and a set of activities $A$, we identify with $D[A]$ the set of all possible grounding of templates in $D$ w.r.t. $A$, i.e. all the constraints that can be built using the given activities.

%\paragraph{Logs and traces} We assume that a trace $t$ is a \emph{finite} word over the set of activities (i.e.\ $t\in A^*$), and a log is a \emph{finite} set of traces. The consequences of this choice is that we don't consider parallel events and multiple occurrences of the same trace won't affect our discovery process.\todo{ST: this might be relaxed}
\paragraph{Traces and Logs} We assume that a \emph{Trace} $t$ is a \emph{finite} word over the set of activities (i.e., $t\in A^*$, where $A^*$ is the set of all the words that can be build on the alphabet defined by $A$).
%
Usually a log is defined as a multi-set of traces, thus allowing multiple occurrences of the same trace: the frequency of a certain trace is then considered as an indicator, for example, of the importance of that trace within the process. Since our goal is to learn a (possibly \emph{minimal}) set of constraints able to discriminate between two example classes, we rather opt to consider a \emph{Log} as a \emph{finite set} of traces. As a consequence, multiple occurrences of the same trace will not affect our discovery process.
% Multiple occurrences of the same trace will not affect our discovery process.
%

\paragraph{Declare constraints satisfaction and violation} Recalling the $LTL_f$ semantics \cite{DBLP:journals/tweb/MontaliPACMS10,DBLP:conf/ijcai/GiacomoV13}, and referring to the stadard Declare semantics as in \cite{2008-Pesic}, we say that a constraint $c$ \emph{accepts} a trace $t$, or equivalently that $t$ \emph{satisfies} $c$, if $t \models c$. Similarly, a constraint $c$ rejects a trace $t$, or equivalently $t$ \emph{violates} $c$, if $t \not\models c$. Given that a Declare model $M$ is a conjunction of constraints, it follows that $M$ \emph{accepts} a trace $t$ ($t$ \emph{satisfies} $M$) if $\forall c \in M, t \models c$. Analogously, a model $M$ \emph{rejects} a trace $t$ ($t$ \emph{violates} $M$) if $\exists c \in M, t\not\models c$. In the following, we will write $t \models M$ meaning that M accepts t.


\paragraph{Positive and negative examples} Finally, we respectively denote with $L^+$ and $L^-$ the sets of positive and negative examples (traces), reported in the input event log. We assume that:
\begin{enumerate*}[label=(\textit{\roman*})]
\item $L^+ \cap L^- = \varnothing$, and 
\item for each trace $t \in L^-$ there exists at least one grounded Declare constraint $c \in D[A]$ that accepts all the positive traces and excludes $t$.
\end{enumerate*}
In other words, we assume that the problem is feasible\footnote{Notice that sometimes real cases might not fulfill these assumptions. We will discuss this issue in section \ref{subsec:impl}}.

%\paragraph{Template subsumption} As pointed out by Di Ciccio et al. \cite{2017-DiCiccio} Declare templates can be organised into a subsumption hierarchy according to the logical implications that can be derived from their semantics. Intuitively, as the constraint \textsf{INIT(a)} forces all traces to start with \textsf{a}, this also implies that \textsf{a} must exist in every trace i.e., \textsf{EXISTENCE(a)}. This relation is valid irrespectively of the involved activity. In a sense, we could say that the template \textsf{EXISTENCE(X)} is \emph{more general} than \textsf{INIT(X)}. This idea is frequently expressed through the subsumption operator $\sqsupseteq$. Given two templates $d, d' \in D$, we say that $d$ \emph{subsumes} $d'$, i.e. $d$ \emph{is more general then} $d'$ (written $d\sqsupseteq d'$), if for any grounding of the involved parameters w.r.t. the activities in $A$, whenever a trace $t \in A^*$ is compliant with $d'$, it is also compliant with $d$ \cite{2017-DiCiccio} .

%\todo[inline]{ST: we don't use \emph{template subsumption} but the more general idea of generality of models (see \ref{def:generality}), even the rules are more general. Maybe this is not the right place for this paragraph, we should move it where we talk about the rules or to the related works section.}

%\tododl{aggiungere anche definizione di subsumption?}


