% !TEX root = ../deviant.tex

\section{Conclusion}
\todoindl{recap \& future works... }
Finally, the performance of the technique presented in this paper could be boosted through a parallel approach. Analogously to previous works \cite{2018a-Maggi, 2018-Loreti, 2020b-Loreti} we can envisage two possible directions to decompose our task: by spitting the model (i.e. in this case the set of constraints to be learned), or the input data (i.e. the business log). The algorithm presented here could easily adopt the first kind of partitioning, whereas the second might be more challenging.

% the process mining task: the set of constraints (to be checked for compliance or learned), and the business log. In this regard, the algorithm presented in this work could easily adopt the first kind of partitioning, whereas the second might be more challenging.


%Finally\tododl{come future work}, the performance of our approach could be boosted through a parallel approach as the one presented in \cite{2018-Loreti}---related to compliance checking---and \cite{2018a-Maggi}---focused on process discovery. Both these works envisage two possible directions to decompose the process mining task: the set of constraints (to be checked for compliance or learned), and the business log. In this regard, the algorithm presented in this work could easily adopt the first kind of partitioning, whereas the second might be more challenging.
